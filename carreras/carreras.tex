% !TeX spellcheck = es_ES
\documentclass[a4paper,12pt]{article}
\usepackage[utf8]{inputenc}
\usepackage[spanish]{babel}
\usepackage{url}
\usepackage{lipsum}

%opening
\title{Carreras enfocadas a sistemas complejos}
\author{Barrera Pérez Carlos Tonatihu \\ Profesor: Genaro Juárez Martínez \\ Computing Selected Topics \\ Grupo: 3CM8 }

\begin{document}

\maketitle
\newpage
\tableofcontents
\newpage
\section{Introducción}
En este documento se pretende informar sobre todas las instituciones que ofrecen la licenciatura, maestría o doctorado que estén enfocadas a los sistemas complejos. Es importante señalar que las instituciones que se toman en cuenta son a nivel internacional.

Se incluyeron carreras totalmente enfocadas a sistemas complejos y aquellas que tienen especialidad en dicha disciplina.

\section{Desarrollo}
Las siguientes instituciones son aquellas que de manera especifica ofrecen la disciplina en sistemas complejos.
\subsection{Universidad de Navarra}
La Universidad de Navarra es una universidad española privada sin ánimo de lucro ubicada en Pamplona (Navarra, España). Esta universidad ofrece el doctorado en Sistemas complejos en la Facultad de Ciencias. El doctorado esta enfocado en la investigación y tiene una duración ordinaria de cuatro años y se centra en dos grupos con sus respectivas lineas de investigación:
\begin{enumerate}

 \item Complejidad en fluidos y biofísica
 \begin{itemize}
  \item Biofísica y física médica
  \item Patrones en convección, evaporación y condensación
  \item Transiciones de fase en coloides
  \item Turbulencia y magnetohidrodinámica
 \end{itemize}

 \item Medios Granulares. Fundamentos Matemáticos y Aplicaciones
 \begin{itemize}
  \item Atascos y desatascos
  \item Estádistica y dinámica de medios granulares
  \item Grupos topológicos y aplicaciones de la geometría y la topología
  \item Softcomputing y aplicaciones
 \end{itemize}

\end{enumerate}

\subsection{Universidad Politécnica de Madrid}
La Universidad Politécnica de Madrid (UPM) es una universidad pública con sede en la Ciudad Universitaria de Madrid (España) cuenta con el doctorado en Sistemas Complejos y cuenta con seis lineas de investigación.
\begin{enumerate}
 \item Caos clásico y cuántico
 \item Automatas celulares
 \item Crecimiento fractal en biofísica
 \item Redes complejas
 \item Dinámica de ondas no lineales en óptica
 \item Efectos multisensoriales en neurología
\end{enumerate}

Además, la UPM ofrece la maestría en Física de Sistemas Complejos.

\subsection{Universidad Adolfo Ibáñez}
La Universidad Adolfo Ibáñez es una universidad privada chilena, dicha universidad cueenta con la mestría y doctorado en Ingeniería de Sistemas Complejos.
\subsection{Universidad Autónoma de la Ciudad de México}
La Universidad Autónoma de la Ciudad de México ofrece la maestría en ciencias de la complejidad y se enfoca en dos lineas de generación con sus respectivos proyectos de investigación
\begin{enumerate}
 \item Sociocomplejidad
 \begin{itemize}
  \item Sociocomplejidad y simulación computacional de dinámicas sociales
  \item Dinámica de la economía y los mercados financieros
  \item Aplicaciones de la teoría de los sistemas complejos a la organización de los servicios públicos y la solución de problemas urbanos
 \end{itemize}

 \item Biología Teórica
 \begin{itemize}
  \item Dinámicas en conflicto, frustración y emergencia de patrones
  \item Morfogénesis y evolución biológicas
  \item Biología teórica
  \item Genómica computacional y reconocimiento de patrones
  \item Dinámica de medios excitables
  \item Epidemiología, inmunología y ecología matemáticas
  \item Modelos matemáticos de la evolución de genes y proteínas
  \item Máquinas moleculares biológicas
  \item Medioambiente y complejidad
 \end{itemize}

\end{enumerate}

\subsection{University of Vermont Complex  Systems Center}
Esta universidad ofrece la maestria y el doctorado en sistemas complejos y ciencia de los datos, la maestría tiene una duración de dos años, por otro lado el doctorado se empezara a impartir apartir del 2018.
\subsection{Universitat de les Illes Balears}
Esta universidad en conjunto con el Consejo Nacional de Investigación de España ofrecen la maestria en física de sistemas complejos. 
\subsection{Polytechnic University of Turin}
Esta universidad ofrece la maestría en Física de sistemas complejos y tiene una duración de dos años.
\subsection{The University of Sydney}
Esta universidad ofrese la maestría en sistemas complejos, tiene una duracion de 2 a cuatro años y se imparte en la facultad de ingeniería y tecnologías de la información de dicha universidad. 
\subsection{Indiana University Bloomington}
En la facultad de informatica, computación e ingeniería de esta universidad se imparte el doctorado de redes y sistemas complejos, dicho doctorado se imparte junto a la Indiana University Network Science Institute

\nocite{navarra}
\nocite{madrid}
\nocite{uno}
\nocite{dos}
\nocite{tres}
\nocite{cuatro}
\nocite{cinco}
\nocite{seis}

\bibliography{carreras}
\bibliographystyle{ieeetr}
\end{document}
