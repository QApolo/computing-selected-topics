% !TeX spellcheck = es_ES
\documentclass[a4paper, 12pt]{article}
\usepackage[utf8]{inputenc}
\usepackage[spanish]{babel}
\usepackage{url}
\usepackage{lipsum}

%opening
\title{Conferencias Macy}
\author{Barrera Pérez Carlos Tonatihu \\ Profesor: Genaro Juárez Martínez \\ Computing Selected Topics \\ Grupo: 3CM8 }


\begin{document}

\maketitle
\newpage
\section{Introducción}
Las Conferencias Macy fueron una serie de reuniones de estudiosos de diversas disciplinas, celebradas en Nueva York por iniciativa de Warren McCulloch y la Fundación Macy de 1946 a 1953. \cite{WEB}

Fue uno de los primeros eventos organizados juntando varias disciplinas, con el objetivo de generar avances en la teoría de sistemas, la cibernética, entre otras ramas.

Entre los asistentes a dichas conferencias estan algunos de los fundadores de la era computacional tales como John von Neumann y Claude Shannon.

Muchas de las discuciones que sucedieron en dichas conferencias se convirtieron despues en ramas de estudio cientifico y funcionaron como bases para dichas disciplinas. 

\section{Conferencias}
Diversos temas fueron discutidos a lo largo de las diez conferencias, a continuacion se presentan dichos temas con un poco de contexto para cada conferencia.
\subsection{Primera conferencia}
La primera de estas conferencias celebrada del 21 al 22 de marzo de 1946 fue enfocada a los mecanismos de retroalimentación y sistemas causales circulares en sistemas biológicos y sociales. Entro los temas que se comentaron se encuentran:
\begin{itemize}
 \item Mecanismos autorreguladores y teológicos
 \item Redes neuronales simulando el calculo de logica proposicional
 \item Antropologia y como las computadoras podrian aprender a como aprender
 \item Mecanismos de retroalimentación de percepción de objetos
 \item Derivando la ética de la ciencia
 \item Comportamiento repetitivo compulsivo
\end{itemize}
\subsection{Segunda conferencia}
El tema de la segunda conferencia fue infuenciado por la publicacion del articulo \textit{Behavior, Purpose and Teleology} por Rosenblueth, Wiener y Bigelow en 1943. La segunda conferencia ocurrio en Octubre de 1946 y fue titulada \textit{Mecanismos Teleológicos y Sistemas Causales Circulares}. Los temas tocados en esa reunion fueron los siguientes:
\begin{itemize}
\item Mecanismos teleológicos en la sociedad
\item Conceptos de la psicología Gestalt
\item Comunicaciones táctiles y químicas entre soldados de hormigas
\end{itemize}

\subsection{Tercera conferencia}
En marzo de 1947 se llevo acabo la tercera conferencia y de nuevo se titulo  \textit{Mecanismos Teleológicos y Sistemas Causales Circulares}. El tema de la reunion fue:
\begin{itemize}
 \item Psicología de niños
\end{itemize}

\subsection{Cuarta conferencia}
El titulo para la cuarta conferencia de octuble de 1947 fue \textit{Causal circular y mecanismos de retroalimentación en los sistemas biológicos y sociales} similar a la primera conferencia. Los temas fueron los siguientes:
\begin{itemize}
 \item La perspectiva de la psicología
 \item Aproximaciones analógicas y digitales de modelos psicológicos
\end{itemize}

\subsection{Quinta conferencia}
La quinta conferencia, primavera de 1948, mantuvo el titulo de la anterior
\begin{itemize}
 \item Formación de la ``I'' en el lenguaje
 \item Modelos formales aplicados al picoteo de los pollos
\end{itemize}

\subsection{Sexta conferencia}
En la sexta conferencia se acordo utilizar el nombre de \textit{Cybernetics} y los titulos de las conferencias anteriores pasaron a ser subtitulos
\begin{itemize}
 \item Memoria
 \item ¿Tenemos neuronas y conexiones suficientes para el total de las capacidades
humanas?
\item Colaboración entre física y psicología
\end{itemize}

\subsection{Septima conferencia}
\begin{itemize}
 \item Interpretación analógica y digital de la mente
 \item Lenguaje y teoría de la información
 \item Lenguaje, símbolos y neurosis
 \item Comprensión de las comunicaciones verbales
 \item Análisis formal de la redundancia semántica en el inglés impreso
\end{itemize}

\subsection{Octava conferencia}
\begin{itemize}
 \item Información como semántica
 \item ¿Los autómatas pueden adaptarse a la lógica deductiva?
 \item Teoría de decisiones
 \item Dinámica de grupos pequeños y comunicaciones de grupo
 \item La aplicabilidad de la teoría de juegos a las motivaciones físicas
 \item El tipo de lenguaje necesario para analizar el lenguaje
 \item Comportamiento puro vs comunicación verdadera
 \item ¿La psiquiatría es una ciencia?
 \item ¿Puede un evento mental que crear memoria ser inconsciente?
\end{itemize}

\subsection{Novena conferencia}
\begin{itemize}
 \item La relación de la neuropsicología con la resolución de problemas en la
filosofía y epistemología
\item La relación de la Cibernética a un micro nivel con la biomedicina y los
procesos celulares
\item La complejidad de los organismos con una función de la información
\item Humor, comunicaciones y paradojas
\item ¿Los autómatas ajedrecisticos necesitan ser aleatorios para poder vencer
a los humanos?
\item Homeostasis y aprendizaje
\end{itemize}
\subsection{Decima conferencia}
\begin{itemize}
 \item Estudios de la actividad en el cerebro
 \item Información semántica y sus medidas
 \item Significado en el lenguaje y como es este adquirido
 \item Como los mecanismos neuronales puede reconocer formas y acordes
musicales
\end{itemize}
\bibliography{macy} 
\bibliographystyle{ieeetr}
\end{document}
