\section{Máquina de Turing}
\subsection{Introducción}
La máquina de Turing, presentada por Alan Turing en 1936 en On computable numbers, with an application to the Entscheidungsproblems, es el modelo matemático de un dispositivo que se comporta como un autómata finito y que dispone de una cinta de longitud infinita en la que se pueden leer, escribir o borrar símbolos. Existen otras versiones con varias cintas, deterministas o no, etc., pero todas son equivalentes (respecto a los lenguajes que aceptan).

Uno de los teoremas más importantes sobre las máquinas de Turing es que pueden simular el comportamiento de una computadora (almacenamiento y unidad de control). Por ello, si un problema no puede ser resuelto por una de estas máquinas, entonces tampoco puede ser resuelto por una computadora (problema indecidible, NP) \cite{LIBRO}.

\begin{figure}[H]
\begin{center}
 \includegraphics[width=10cm, height=3cm]{./img/maquina.png}
 \caption{Una máquina de Turing}
 \label{fig:diagrama}
\end{center}
\end{figure}

De manera formal la máquina de Turing se define de la siguiente manera:

\[ M=(Q, \Sigma, \Gamma, \delta, q_0, B, F) \]
donde:
\begin{description}
 \item $Q$ es el conjunto finito de estados de la unidad de control.
 \item $ \Sigma $ es el conjunto finito de símbolos de entrada
 \item $ \Gamma $ es el conjunto completo de símbolos de cinta; $ \Sigma $ siempre es un subconjunto de $\Gamma$
 \item $\delta$ Es la función de transición. Los argumentos de $\delta(q, X)$ son un estado $q$ y un símbolo de cinta $X$. El valor de $\delta(q, X)$, si está definido, es $(p, Y, D)$ donde:
 \begin{enumerate}
  \item $p$ es el siguiente estado de $Q$
  \item $Y$ es el símbolo de $\Gamma$, que se escribe en la casilla que señala la cabeza y que sustituye a cualquier símbolo que se encontrara en ella.
  \item $D$ es una dirección y puede ser $L$ o $ R$, lo que nos indica la dirección en que la cabeza se mueve, `izquierda` (L) o `derecha` (R), respectivamente
 \end{enumerate}
 \item $q_0$ es el estado inicial, un elemento de $Q$, en el que inicialmente se encuentra la unidad de control.
 \item $B$ es el símbolo espacio en blanco. Este símbolo pertenece a $Gamma$ pero no a $Sigma$; es decir, no es un símbolo de entrada.
 \item $F$ es el conjunto de los estados finales, un subconjunto de $Q$
\end{description}

\subsection{Planteamiento de la práctica}
La elaboración de este programa consistió en elaborar un máquina de Turing capaz de duplicar la cantidad de unos en una cadena de unos, es decir, si la cadena que se ingresa es $ 11 $ entonces la cadena de salida sera $ 1111 $.

Es por esto que la máquina sólo aceptara unos en la entrada mientras que los símbolos de la cinta incluirán a la X y la Y. De esta forma la máquina de Turing para este problema se define como \cite{LIBRO}:

\[M=(\lbrace q_{0}, q_{1}, q_{2}, q_{3}\rbrace, \lbrace 1 \rbrace, \lbrace 1, X, Y, B \rbrace, \delta, q_{0}, B, \lbrace q_{4} \rbrace)\]

donde $ \delta $ se especifica en la siguiente tabla:
\begin{center}
\begin{tabular}{|c|c|c|c|c|}
\hline
& \multicolumn{4}{|c|}{Símbolo} \\ \hline
Estado & 1 & X & Y & B\\ \hline
0 & (1, X, R) & - & (3, 1, R) & -\\ \hline
1 & (1, 1, R) & - & (1, Y, R) & (1, Y, L)\\ \hline
2 & (2, 1, L) & (0, 1, R) & (2, Y, L) & -\\ \hline
3 & - & - & (3, 1, R) & - \\ \hline
\end{tabular}
\end{center}

EL funcionamiento de esta máquina se puede entender mejor con el diagrama de la figura \ref{fig:diagrama}

\begin{figure}[H]
\begin{center}
 \includegraphics[width=12cm, height=7cm]{./img/Turing-unos.png}
 \caption{Representación gráfica de las transiciones de la máquina de Turing}
 \label{fig:diagrama}
\end{center}
\end{figure}

El funcionamiento del programa empieza cuando se inserta una cadena de unos y la máquina empezara a trabajar, además de generar una cadena final se muestra una animación del como se están realizando las transiciones en la cinta de la máquina, por otro lado se imprime en consola el historial de movimientos que se hacen, este historial también se guarda en un archivo de texto.

\subsection{Desarrollo}
El código de este programa fue realizado en Python 3.7 y se utilizo la biblioteca tkinter para la parte gráfica.

Archivo: turing.py
Esta clase es la que modela la máquina de Turing, los parámetros importantes son: el estado inicial, los estados finales, la cadena de entrada y la función de transición.
\begin{lstlisting}[language=Python]
class MaquinaTuring:
    """MaquinaTuring"""
    def __init__(self, estado_inicial, estados_finales, cadena, transiciones):
        self.transiciones = transiciones
        self.inicial = estado_inicial
        self.finales = estados_finales
        self.cinta = list(cadena)
        self.estado_actual = self.inicial
        self.apuntador = 0
        self.blanco = "B"
        self.direccion = None

    def consumir(self):
        """Toma un simbolo de la cinta y lo evalua en la funcion de
        transicion"""
        if len(self.cinta) - 1 < self.apuntador:
            caracter = 'B'
        else:
            caracter = self.cinta[self.apuntador]
        tupla = (self.estado_actual, caracter)
        if tupla in self.transiciones:
            siguiente = self.transiciones[tupla]
            if len(self.cinta) - 1 < self.apuntador:
                self.cinta.append(self.blanco)
            if self.apuntador < 0:
                self.cinta.insert(0, self.blanco)
            self.cinta[self.apuntador] = siguiente[1]
            if siguiente[2] == "R":
                self.apuntador += 1
            else:
                self.apuntador -= 1

            self.estado_actual = siguiente[0]
            self.direccion = siguiente[2]
            return True
        else:
            return False

    def es_final(self):
        """Metodo para comprobar si nos encontramos en un estado final"""
        if self.estado_actual in self.finales:
            if len(self.cinta) - 1 < self.apuntador or self.apuntador < 0:
                return True
        return False
\end{lstlisting}

Archivo: diagrama.py
Este archivo implementa la clase MaquinaTuring.py y se declaran los parámetros que se pasaran a este archivo así como la captura de la cadena y la escritura del registro de transiciones en consola y en archivo de texto, sin olvidar la animación de dichas transiciones.
\begin{lstlisting}[language=Python]
import tkinter as tk
import time
from tkinter import font as tkfont
from turing import MaquinaTuring

# Tabla de transiciones que modela el automata
transiciones = {
            ("q0", "1"): ("q1", "X", "R"),
            ("q0", "Y"): ("q3", "1", "R"),
            ("q1", "1"): ("q1", "1", "R"),
            ("q1", "Y"): ("q1", "Y", "R"),
            ("q1", "B"): ("q2", "Y", "L"),
            ("q2", "1"): ("q2", "1", "L"),
            ("q2", "X"): ("q0", "1", "R"),
            ("q2", "Y"): ("q2", "Y", "L"),
            ("q3", "Y"): ("q3", "1", "R"),
            }

entrada = input("Ingrese la cadena de unos: ")
maquina = MaquinaTuring("q0", "q3", entrada, transiciones)

# Configuracion de la ventana
gui = tk.Tk()
gui.geometry("600x400+100+100")
gui.title("Maquina de Turing")
c = tk.Canvas(gui, width=600, height=400)
c.pack()
bold_font = tkfont.Font(family="Arial", size=24)

# Principales componentes que se animan
control = c.create_rectangle(150, 100, 200, 150, fill="lightblue")
flecha = c.create_line(175, 150, 175, 175, arrow=tk.LAST, width=3)
texto = c.create_text(165, 200, text=''.join(maquina.cinta), font=bold_font,
                      anchor=tk.W)
estado = c.create_text(160, 125, text=maquina.estado_actual, font=bold_font,
                       anchor=tk.W)

archivo = open("salida.txt", "w+")
# Mientras no llegues a un estado final continua
while not maquina.es_final():
    print('Cadena: {}'.format(''.join(maquina.cinta)))
    print('Estado actual: {}, apuntador: {}'.format(maquina.estado_actual,
          maquina.apuntador+1))

    archivo.write('Cadena: {}\n'.format(''.join(maquina.cinta)))
    archivo.write('Estado actual: {}, apuntador: {}\n'
                  .format(maquina.estado_actual, maquina.apuntador+1))
    if not maquina.consumir():
        print('*' * 20)
        archivo.write('*' * 20)
        archivo.write('\n')
        break
    print('Siguiente estado: {}'.format(maquina.estado_actual))
    print('*'*20)

    archivo.write('Siguiente estado: {}\n'.format(maquina.estado_actual))
    archivo.write('*' * 20)
    archivo.write('\n')

    gui.update()
    time.sleep(1)
    c.itemconfigure(texto, text=''.join(maquina.cinta), anchor=tk.W)
    c.itemconfigure(estado, text=maquina.estado_actual)
    if maquina.direccion == 'R':
        c.move(control, 19, 0)
        c.move(flecha, 19, 0)
        c.move(estado, 19, 0)
    else:
        c.move(control, -19, 0)
        c.move(estado, -19, 0)
        c.move(flecha, -19, 0)

print('Cadena final: {}'.format(''.join(maquina.cinta)))
archivo.write('Cadena final: {}\n'.format(''.join(maquina.cinta)))
archivo.close()
gui.mainloop()
\end{lstlisting}

\subsection{Pruebas}
El siguiente ejemplo es la cadena con un solo uno.

\begin{figure}[H]
\begin{center}
 \includegraphics[width=13cm, height=8cm]{./img/uno_consola.png}
 \caption{Salida en consola}
 \label{fig:uno_consola}
\end{center}
\end{figure}

\begin{figure}[H]
\begin{center}
 \includegraphics[width=13cm, height=8cm]{./img/uno_historial.png}
 \caption{Registro de transiciones}
 \label{fig:uno_historial}
\end{center}
\end{figure}

\begin{figure}[H]
\begin{center}
 \includegraphics[width=13cm, height=6cm]{./img/uno_grafica.png}
 \caption{Representación gráfica de las transiciones de la máquina de Turing}
 \label{fig:uno_grafica}
\end{center}
\end{figure}

En este ejemplo se inserta una cadena con dos unos.
\begin{figure}[H]
\begin{center}
 \includegraphics[width=13cm, height=20cm]{./img/dos_consola.png}
 \caption{Salida en consola parte uno}
 \label{fig:dos_consola}
\end{center}
\end{figure}

\begin{figure}[H]
\begin{center}
 \includegraphics[width=13cm, height=7cm]{./img/dos_console.png}
 \caption{Salida en consola parte dos}
 \label{fig:dos_console}
\end{center}
\end{figure}

\begin{figure}[H]
\begin{center}
 \includegraphics[width=13cm, height=20cm]{./img/dos_historial.png}
 \caption{Registro de transiciones}
 \label{fig:dos_historial}
\end{center}
\end{figure}

\begin{figure}[H]
\begin{center}
 \includegraphics[width=13cm, height=6cm]{./img/dos_grafica.png}
 \caption{Representación gráfica de las transiciones de la máquina de Turing}
 \label{fig:dos_grafica}
\end{center}
\end{figure}

\subsection{Conclusiones}
El hacer este programa probó lo útil y poderosa que es la máquina de Turing en la solución de problemas, ya que como es evidente si un problema no se puede solucionar usando una máquina de Turing entonces es no computable y a pesar que este problema de duplicar una cadena no es difícil el implementar una máquina de Turing resulta en una solución bastante practica y fácil de implementar.
